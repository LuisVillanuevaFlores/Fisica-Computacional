\documentclass{article}
\usepackage[utf8]{inputenc}
\usepackage{graphicx}
\usepackage{ragged2e}
\usepackage{amssymb, amsmath, amsbsy}
\usepackage{float}
\usepackage{listings}
\usepackage{xcolor}
\usepackage[spanish]{babel}
\title{kd tree}
\author{Luis Villanueva  }
\date{Nomvember 2019}
\justifying
\begin{document}

%-------COLORES PARA CODIGO ------------------------
\lstdefinestyle{customc}{
  belowcaptionskip=1\baselineskip,
  breaklines=true,
  frame=L,
  xleftmargin=\parindent,
  language=JavaScript,
   basicstyle=\footnotesize\ttfamily,
  showstringspaces=false,
  basicstyle=\footnotesize\ttfamily,
  keywordstyle=\bfseries\color{green!40!black},
  commentstyle=\itshape\color{purple!40!black},
  identifierstyle=\color{blue},
  stringstyle=\color{orange},
  frame=single,	  
  numbers=left,
   numberstyle=\footnotesize,
}

\lstdefinestyle{customasm}{
  belowcaptionskip=1\baselineskip,
  frame=L,
  xleftmargin=\parindent,
  language=JavaScript,
  basicstyle=\footnotesize\ttfamily,
  commentstyle=\itshape\color{purple!40!black},
}

\lstset{escapechar=@,style=customc}

%---------------------------------------------

\thispagestyle{empty}

\vfill
 \begin{center}
    \begin{figure}[h]
    \centering
    \includegraphics[width=12cm]{unsa}\\
    
    \end{figure}
 	 
     \vspace*{1.5cm}
    {\large\bfseries FACULTAD DE PRODUCCIÓN Y SERVICIOS} \\
    {\large\bfseries ESCUELA PROFESIONAL DE CIENCIA DE LA COMPUTACIÓN}  \\ 
    
    \vspace*{1 cm}
      {\large\bfseries Práctica 6  }
       \vspace*{0.5 cm}
      
    
 	\rule[0.5ex]{\linewidth}{2pt}\vspace*{-\baselineskip}\vspace*		{3.2pt}
	\rule[0.5ex]{\linewidth}{1pt}\\[\baselineskip]
 	{\huge Física Computacional} \\[4mm]
    \rule[0.5ex]{\linewidth}{1pt}\vspace*{-							\baselineskip}\vspace{3.2pt}
	\rule[0.5ex]{\linewidth}{2pt}\\
  \vspace*{1 cm}

    \begin{large}
    \bfseries DOCENTE: \\
    \bfseries Edwin Llamoca Requena  \\
     \vspace{5mm}
    \bfseries ALUMNO:\\
    \bfseries Luis Guillermo Villanueva Flores \\
   
   
    \end{large}
    \vspace*{0.4in}
    \noindent \\
    
    \vfill
    \large\bfseries{ AREQUIPA\\2020}
\end{center}
\newpage


\begin{flushleft}
  

\section{Solución aproximada de la ecuación de Onda}

\begin{enumerate}
    \item Analice el mismo problema, cambiando h y k y verifique sus resultados.
    \vspace{0.2 cm}
    Para que en el problema se mantenga y cumpla con la condición de estabilidad, se cambio a h a $0.08$ y a k como $0.03$, de esa forma se obtiene, un r con valor de 0.75, de esa forma se mantiene la condición de estabilidad, el código se hizo de la siguiente forma, tal como se presentó en clase:
    
    \begin{lstlisting}[language=Python,caption=Código para ejercicio 1]
function U=onda(f,g,a,b,v,h,k)
n=a/h+1;
m=b/k+1;
r=v*k/h;
r
r1=r^2;
r2=r^2/2;
s1=1-r^2;
s2=2*(1-r^2);
U=zeros(n,m);
for i=2:n-1
  U(i,1)=feval(f,h*(i-1));
  U(i,2)=s1*feval(f,h*(i-1))+k*feval(g,h*(i-1))+r2*(feval(f,h*i)+feval(f,h*(i-2)));
end

for j=2:m-1
  for i=2:n-1
      U(i,j+1)= s2*U(i,j)+r1*(U(i-1,j)+U(i+1,j))-U(i,j-1);
  end
  surfc(U)
end


\end{lstlisting}


Hay que ver que para este problema, se usa dos funciones, f y g, que se definen en código de la siguiente manera:

   \begin{lstlisting}[language=Python,caption=Función f]
function y=f(x)
  y=x^2-x+sin(2*pi*x)
endfunction
\end{lstlisting}

  \begin{lstlisting}[language=Python,caption=Función g]
function y=g(a)
  y=0;
endfunction
\end{lstlisting}

Al ejecutar este código la gráfica queda de la siguiente manera, hay que resaltar que se cambiaron los valores de h y k, para este problema, pero que cumpla con la condición de estabilidad, los valore cambiados se detallan en la parte superior:

\begin{figure}[H]
        \centering
         \includegraphics[width=1.1\textwidth]{1.png}
        \caption{Resultado}
\end{figure}   

Se puede apreciar la correspondiente gráfica.

\newpage
\item Verifique que si no cumple $r\leq 1$, como se verán los resultados?\\
 \vspace{0.2 cm}
Para este problema,se cambio los valores de h y k a 0.1 y 0.2 respectivamente, de esa forma podemos darnos cuenta que no cumple la condición de estabilidad, ya que r nos saldría 4, por ende obtenemos esta gráfica:
\begin{figure}[H]
        \centering
         \includegraphics[width=1.1\textwidth]{2.png}
        \caption{Resultado}
\end{figure} 
\end{enumerate}
\vspace{0.3 cm}
\section{Problema desafío}
\begin{enumerate}
    \item Use  Use a = 1, b = 1, v = 1, $g(x) = 0$,\\
  \vspace{0.5 cm}
  $  f(x)= \left\{ \begin{array}{lcc}
             2x &   si  & 0 \leq x \leq 1/2 \\
             \\ 2-2x &  si & 1/2 \leq x \leq 1 \\
            
             \end{array}
   \right.
   $
   \\
  
  \newpage
  
   \vspace{0.5 cm}
   Para este problema solo se cambio la función f que nos siguiere este problema, para esto se realizó en código de la siguiente manera:
   
    \begin{lstlisting}[language=Python,caption=Función f]
function y=f2x(x)
  if (x>=0 && x<=0.5)
    y=2*x;
  end
  if (x>=0.5 && x<=1)
    y=2-2*x;
  end
endfunction

\end{lstlisting}
   
Ahora si realizamos un surf a partir de esta función, podemos apreciar la siguiente gráfica:
\begin{figure}[H]
        \centering
         \includegraphics[width=1.1\textwidth]{3.png}
        \caption{Resultado}
\end{figure} 

Ahora si le damos otra perspectiva a esta gráfica, o sea cambiarle de orientación podríamos obtener lo siguiente:
\begin{figure}[H]
        \centering
         \includegraphics[width=1.1\textwidth]{4.png}
        \caption{Resultado}
\end{figure} 

Ahora, también se puede tener otra perspectiva, como cuando la onda llega a 0 y de ahí se vuelve a levantar, se podría apreciar gráficamente, algo así:

\begin{figure}[H]
        \centering
         \includegraphics[width=1.1\textwidth]{5.png}
        \caption{Resultado}
\end{figure} 

En la gráfica anterior se puede ver una comparativa, de como es cuando llego a 0 y empieza a elevarse otra vez, de esa forma podemos hacer una comparativa, entre el movimiento que se genero, cuando estaba descendiendo, y con el que se estaba creciendo nuevamente.

\vspace{0.4 cm}
\item Haga una secuencia de evolución paso a paso y que se grafique en forma independiente, ejemplo,
utilice la instrucción subplot(10,10,x). Subplot(10,10,1) para j=1, Subplot(10,10,2) para j=2,
etc.
 
 \vspace{0.2 cm}
 Para este problema se obtuvo las siguientes gráficas, para unos 30 momentos de nuestra matriz:
 \begin{figure}[H]
        \centering
         \includegraphics[width=1.1\textwidth]{6.png}
        \caption{Resultado}
\end{figure} 

Si agrandamos un poco más el número de visualizaciones por cada fila de nuestra matriz, podemos obtener una mejor perspectiva:
Para este problema se obtuvo las siguientes gráficas, para unos 30 momentos de nuestra matriz:
 \begin{figure}[H]
        \centering
         \includegraphics[width=1.1\textwidth]{7.png}
        \caption{Resultado}
\end{figure} 
 
 
\end{enumerate}

 



\end{flushleft}
\end{document}
